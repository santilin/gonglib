\chapter{Módulo de gestión de los contactos}

El módulo de gestión de los contactos proporciona una base de datos
personales global y única para toda la organización: personal
contratado, personas socias, voluntarias, acreedoras, etc. o cualquier
otra persona de interés para la organización.

Con este módulo se puede:

\liststyleLvii
\begin{itemize}
\item Añadir y eliminar contactos mecánicamente.
\item Importar y exportar masivamente listas de contactos.
\end{itemize}
Para mantener la base de datos de contactos, ve a
\textstyleGUIELEMENT{Asociación $\rightarrow $ Contactos...}

\section{El fichero de datos personales de los contactos}
Los datos de contacto de las personas físicas o jurídicas de vuestra
organización se almacenan en un fichero aparte del fichero de
miembros de los proyectos. Las ventajas de este diseño son:

\liststyleLviii
\begin{enumerate}
\item Se pueden reutilizar los contactos evitando duplicarlos si tenemos
una persona que es a la vez socia, empleada, acreedora, ...,
\item Toda la información de carácter personal está recogida en un
solo fichero, lo cual es útil para el cumplimento de la LOPD (Ley
Orgánica de Protección de Datos de Carácter Personal), 
\item Si un dato personal cambia, solo hará falta cambiarlo en un
lugar.
\item Se pueden introducir en el sistema contactos que no son socias o
que no tienen ninguna relación administrativa con la organización
\end{enumerate}

\bigskip

