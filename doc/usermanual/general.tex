\chapter{Bienvenida}
\section{¿Qué es \appname?}

\appname es parte de ProgramacionSocial.net.


GestiONG es una filosofía.


\appname es un programa para la \appdescription desarrollado bajo el paradigma del
software libre con el objetivo de ser utilizado en sistemas operativos
libres como GNU/Linux. La versión actual permite la gestión de la
base social: contactos, altas y bajas, cuotas, \ldots; la gestión
contable adaptada al nuevo plan del 2008 y una completa gestión
económica de proyectos, incluyendo remesas de recibos y gastos e
ingresos por partidas.

\section{¿Cuál es el modelo de negocio de \appname?}

\appname es libre y la base general del software es además gratuita. 
El modelo de negocio consiste en dos vías:
\begin{itemize}
 \item Cobrar por el trabajo efectivo de instalación, puesta en marcha y formación al cliente.
 \item Cobrar por el desarrollo de módulos específicos para cada cliente.
\end{itemize}


\section{¿A quién va dirigido \appname?}
\appname permite llevar la contabilidad de una organización sin
ánimo de lucro a personas no expertas en informática ni en
contabilidad: tesoreras, secretarias e incluso voluntarias. Se ha
realizado un gran esfuerzo, por un lado, para que el manejo del
programa sea fácil, intuitivo y seguro sin necesidad de saber
informática y por otro lado, para llevar la contabilidad oficial sin
necesidad de saber contabilidad.

\appname es, además, un proyecto posicionado firmemente en cuestiones
que atañen a la libertad de uso, al tipo de organizaciones a las que
va destinado y al uso del lenguaje de los manuales y de los textos del
programa.

En cuanto a la libertad, todo el proyecto, incluyendo diseño,
código, documentación e imágenes está distribuido bajo una
serie de licencias conocidas genéricamente como copyleft. Estas
licencias promueven la distribución, uso y modificación libre de
cualquier parte del proyecto respetando los derechos de autoría e
intentando evitar su apropiación por parte de terceras personas. El
código está licenciado bajo la Licencia Pública General GNU (GPL,
GNU General Public Licence) y la documentación bajo la Licencia de
Documentación Libre de GNU (GNU Free Public License). El compromiso
de \appname con el software libre llega hasta el punto de no apoyar su
empleo en sistemas operativos propietarios ni formatos de documentos
que no sean libres. Esta decisión de diseño ha retrasado, quizás,
la adopción del proyecto por parte de organizaciones que aún no
estaban preparadas para el cambio efectivo de mentalidad, pero aún
así, \appname se ha mantenido, y espera mantenerse en el futuro, fiel
a sus principios.

En cuanto al tipo de organizaciones a las que va destinado, \appname se
posiciona a favor de las organizaciones sin ánimo de lucro: ONG,
asociaciones y clubes de todo tipo, cooperativas de trabajo, centros
sociales, organismos públicos, etc. En este sentido, las necesidades
de organizaciones con ánimo de lucro serán simplemente desoídas.

Por último, en cuanto al lenguaje empleado, se ha preferido el uso del
femenino genérico no solo en el programa y en la documentación,
sino también en el propio código fuente, lo cual ha sido un
desafío que nos ha mostrado hasta qué punto está arraigado el
lenguaje sexista en las carreras técnicas.

\section{¿Qué se puede hacer con \appname?}
Si bien es cierto que la apariencia de \appname es simple, por dentro
está diseñado para crecer. La mayor parte del esfuerzo de
programación se ha empleado en crear una base que pueda ser extendida
por cada organización; por eso, esta primera versión de
producción ofrece una funcionalidad básica para la gestión
contable, de socias y de proyectos a la espera de conocer las
necesidades reales de las asociaciones. 

\section{Sensibilidades}

\appname es un proyecto dentro de ProgramacionSocial.net. Esto le confiere
una serie de sensibilidades:

\begin{itemize}
 \item Traer la potencia y eficacia de la gestión empresarial a las organizaciones sociales.
 \item Utilizar el femenino genérico.
 \item Apuesta comprometida por el software libre.
 \item Apuesta por lo local.
\end{itemize}


La versión actual permite:

\liststyleLi
\begin{itemize}
\item Puesta en marcha inmediata:

\begin{itemize}
\item Importar los datos de las personas socias desde cualquier hoja de
cálculo.
\item Importar \ cuentas y asientos desde otros programas de
contabilidad.
\end{itemize}
\end{itemize}
\liststyleLii
\begin{itemize}
\item Gestión de las personas socias de la asociación:

\begin{itemize}
\item Dar de alta los datos de contacto y mantener las altas y bajas de
las personas socias.
\item Definir las cuotas y formas de pago de cada persona socia.
\item Producir informes de socias, remesas, recibos, etc.
\item Facilitar el envío masivo de correspondencia y emails.
\end{itemize}
\item Gestión de proyectos:

\begin{itemize}
\item Crear remesas de recibos y enviarlas a las cajas de ahorros y
bancos.
\item Controlar los pagos, impagos y devoluciones de los recibos.
\item Definir las diferentes partidas de gastos e ingresos para cada
proyecto.
\item Asignar cada apunte contable a una partida determinada de un
proyecto.
\item Producir informes de gastos e ingresos por partidas para cada
proyecto.
\item Añadir miembros a los proyectos, con su tipo de filiación y
forma de pago y generar remesas del mismo modo que con las personas
socias.
\end{itemize}
\item Gestión contable:

\begin{itemize}
\item Crear automáticamente el plan contable estándar del 2008 o el
del 1998.
\item Introducir asientos manualmente y a partir de asientos tipo.
\item Producir informes de sumas y saldos, balances, extractos, etc.
\end{itemize}
\item General:

\begin{itemize}
\item Exportar cualquier fichero para: hojas de cálculo, combinación
de correspondencia, envíos masivos de correos electrónicos, etc.
\item Modificar y añadir informes nuevos.
\item Generar informes en PDF, OpenOffice, HTML, etc.
\end{itemize}
\end{itemize}

\bigskip

\section{Características de \appname}

\begin{description}
 \item[Facilidad de instalación y puesta en marcha.]
\appname no presupone ningún conocimiento sobre informática por
parte de la usuaria. Las instrucciones para la instalación y puesta
en marcha son lo suficientemente detalladas y precisas para que el
programa se pueda descargar de Internet e instalar sin problemas. Una
vez instalado, el programa se encarga de crear la base de datos. Una
vez puesto en marcha, hay numerosas opciones para importar los datos de
vuestra organización y comenzar a trabajar inmediatamente.

\item[Facilidad de uso.]
La aplicación permite trabajar con varios ficheros y documentos
simultáneamente. La introducción de datos se ha diseñado para que
sea lo más cómoda y correcta posible, pensando en todo momento en
agilizar el proceso mediante un uso eficientemente el teclado. Todos
los ficheros presentan un aspecto y comportamiento similar con una gran
cantidad de opciones para búsqueda, filtro, ordenación,
impresión, duplicación, importación, exportación, etc.

\item[Simplicidad.]

\appname reduce al máximo la cantidad de datos que la usuaria debe
introducir y maximiza la información que se obtiene de ellos. Así,
cada vez que se introduce un gasto o un ingreso, el programa genera
automáticamente el asiento contable correspondiente. Además, si se
indica la partida a la que pertenece el gasto o ingreso, se mantienen
actualizadas las partidas de gastos e ingresos del proyecto. Cuando se
paga un recibo, además de generarse el asiento contable
correspondiente, se puede imprimir inmediatamente el \textit{recibí}
para la persona socia.


\item[Adaptabilidad.]
\appname puede ser configurado en diversos aspectos para adaptarlo a
cualquier tipo de asociación u organización. 

\liststyleLiii
\begin{itemize}
\item Puede ser usado en cualquier idioma, con cualquier formato de
moneda y fecha.
\item Permite modificar las descripciones de los ficheros y \ campos de
la base de datos para que se ajusten a la nomenclatura de cada
organización.
\item Soporte para múltiples asociaciones, múltiples ejercicios y
múltiples usuarias.
\item Permite la modificación y creación de nuevos informes.
\end{itemize}

\item[Extensibilidad.]
La versión básica posee tres módulos: gestión de socias,
gestión de proyectos y contabilidad. Pero esta base la puede extender
cualquier programadora para desarrollar nuevos módulos o adaptar los
ya existentes. Hasta la fecha, se ha desarrollado un módulo de
gestión de atenciones para una asociación de apoyo a las
trabajadoras del sexo y se está desarrollando un módulo para
inscripciones a talleres y cursos y otro módulo para la gestión de
una biblioteca/mediateca.


\item[Independencia.]
La adhesión al software libre y a los estándares abiertos asegura
que vuestros datos estarán controlados siempre por vuestra
organización y que podréis, llegado el caso, migrarlos a cualquier
otro formato sin ningún tipo de problemas.

\end{description}



\section{Organización de este manual}
El capítulo II comienza con tres tutoriales con instrucciones muy
detalladas y profusamente ilustradas: instalación; puesta en marcha y
comienzo del trabajo efectivo. El objetivo es poner en marcha \appname
en vuestra organización independientemente de los conocimientos
informáticos que poseáis.

Los siguientes capítulos, necesariamente incompletos aún, realizan
un somero análisis de cada módulo del sistema: III Gestión de las
personas socias, IV Gestión de proyectos y \ V Gestión contable.

Los últimos capítulos contienen información sobre otros aspectos
de \appname que permitirán un uso más efectivo del mismo: VI)
Configuración; VII) Generación de informes; VIII) Herramientas del
sistema.



