\chapter{Personalización}
\section{Idioma de la aplicación}
\appname está preparado para mostrar sus mensajes en el idioma que
elija la usuaria. Los idiomas disponibles hasta ahora son: español,
catalán, gallego y euskera, aunque no todos están completos.

El idioma viene predeterminado normalmente en la configuración del
escritorio que utilizas. Si quieres cambiarlo:

a) Si usas KDE, ve a \textstyleGUIELEMENT{Menú K
}\textstyleGUIELEMENT{$\rightarrow $}\textstyleGUIELEMENT{ Centro de
control }\textstyleGUIELEMENT{$\rightarrow $}\textstyleGUIELEMENT{
Regional y accesibilidad }\textstyleGUIELEMENT{$\rightarrow
$}\textstyleGUIELEMENT{ País/Región e idioma}.

\section{Ficheros de configuración globales y locales}
\section{Diccionario de datos: nombres de ficheros y campos }
\appname posee un diccionario de datos con información sobre los
distintos elementos de la base de datos, por ejemplo, los títulos de
las tablas, los tipos y descripciones de los campos, su tamaño, su
estilo, su valor por defecto, etc.

Algunos de los elementos del diccionario de datos se pueden personalizar
modificando unos ficheros de texto con extensión .ddd (Diccionario De
Datos) que se leen al iniciar la aplicación y que se encuentran en el
subdirectorio \textstylenombreprograma{database} del directorio de
configuración general o local. Al modificar esta información, los
cambios se verán reflejados en los menús, formularios e informes de
la aplicación.

Suele haber un fichero por cada módulo de \appname. Así, para el
módulo de socias, el fichero se llama socias.ddd, para el de
contabilidad, conta.ddd y así sucesivamente. Además, se pueden
crear ficheros .ddd para cada idioma que se desee. Basta con anteponer
el código del lenguaje entre guiones bajos al nombre del fichero: por
ejemplo, \_es\_socias.ddd, \_ca\_conta.ddd, etc.

El contenido de esos ficheros es como sigue:



\begin{center}
\begin{minipage}{16.249cm}
\textstyleUserEntry{ASOCIACION.DESC\_SINGULAR=Asociación}

\textstyleUserEntry{ASOCIACION.DESC\_PLURAL=Asociaciones}

\textstyleUserEntry{ASOCIACION.FEMENINA=true}


\bigskip

\textstyleUserEntry{ASOCIACION.CODIGO.CAPTION=Código}

\textstyleUserEntry{ASOCIACION.CODIGO.DESCRIPTION=Código}

\textstyleUserEntry{ASOCIACION.CODIGO.STYLE=CODE}

\textstyleUserEntry{ASOCIACION.CODIGO.WIDTH=0}

\textstyleUserEntry{ASOCIACION.CODIGO.DEFAULTVALUE=}

\textstyleUserEntry{ASOCIACION.NOMBRE.CAPTION=Nombre abreviado
asociación}
\end{minipage}
\end{center}

\bigskip


\bigskip


\bigskip


\bigskip


\bigskip


\bigskip


\bigskip


\bigskip


\bigskip


\bigskip

Las primeras tres líneas definen la tabla de asociaciones, que en la
base de datos se llama
{\textquotedblleft}ASOCIACION{\textquotedblright}. El nombre en
singular de la tabla que aparecerá en \appname será
{\textquotedblleft}Asociación{\textquotedblright}, en plural
{\textquotedblleft}Asociaciones{\textquotedblright} y se declara que el
nombre es femenino. Se podría cambiar
{\textquotedblleft}Asociación{\textquotedblright} por
{\textquotedblleft}Organización{\textquotedblright} y el cambio se
vería reflejado en los menús, formularios e informes.

Las siguientes líneas definen los campos de la tabla de asociaciones.
El campo CODIGO llevará por título
{\textquotedblleft}Código{\textquotedblright}, su descripción
será también {\textquotedblleft}Código{\textquotedblright}, su
estilo {\textquotedblleft}Code{\textquotedblright} y la longitud y el
valor por defecto se dejan sin tocar. A continuación se define el
campo NOMBRE, y así sucesivamente. Un valor muy útil del
diccionario de datos es \textstyleUserEntry{DEFAULTVALUE}, que es el
valor que tomará un campo al ser creado, evitando de este modo tener
que introducirlo cada vez que se da de alta un registro. Por ejemplo,
si ponemos \textstyleUserEntry{CONTACTO.PAIS.DEFAULTVALUE=España},
cada vez que demos de alta un contacto, su país será España.

Inicialmente, \appname instala en el directorio de configuración
global los diccionarios de datos por defecto para diversos lenguajes.
Estos ficheros solo puede modificarlos la superusuaria, por lo que para
cambiarlos, tienes que crear una copia en tu directorio de
configuración local.
