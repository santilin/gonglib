\chapter{Carga masiva de datos}

La carga masiva de datos de los artículos, familias, clientes, proveedoras, etc. se realiza de una forma sencilla mediante  una serie de plantillas de hojas de cálculo que se han instalado en tu ordenador y que pueden ser rellenadas con un programa de hoja de cálculo como \commercialname{LibreOffice Calc}, \commercialname{GNumeric} u \commercialname{OpenOffice Calc}.

Las plantillas se encuentran en el directorio \directorio{factu/plantillas/importacion} en el directorio global de datos de \appname\footnote{Puedes ver cuál es este directorio en la opción del menú \menu{Ayuda=>Acerca de...}}. Ejecuta tu programa de hoja de cálculo favorito y abre la plantilla que quieras rellenar: \filename{PROVEEDORA.ods}, \filename{CLIENTE.ods}, \filename{FAMILIA.ods} y \filename{ARTICULO.ods}. Antes de nada, guarda una copia en tu escritorio porque las plantillas originales no se pueden modificar. En las propias plantillas tienes las instrucciones para rellenar los datos. 

\hint{Cuanto más completa y precisa sea la información que introduzcas en estas plantillas, más tiempo ahorrarás posteriormente.}

Una vez rellenos los datos guarda la plantilla y ve al fichero de correspondiente (\menu{Facturacion => Familias}, etc.) y elige la opción \menu{Fichero => Importar}, elige la plantilla (\filename{FAMILIA.ods}, etc.) que has guardado en tu escritorio y los datos se añadirán al fichero correspondiente. 

Repite este proceso con los ficheros de proveedoras, clientes y artículos.

\recuadro{Trucos de importación}{%
Las celdas en blanco no serán importadas. Para borrar el valor de un campo, escribir ~ en la celda.
}
